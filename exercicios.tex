\documentclass[12pt]{article}
\usepackage[utf8]{inputenc}
\usepackage{lmodern}
\usepackage[T1]{fontenc}

\title{Exercicios Infraestrutura Computacional }
\author{Karoline Rodrigues}
\begin{document}
\normalsize
\maketitle
\section{Aula 1}
\label{sec:Aula1}
\begin{description}
    \item[1] Quais as diferenças na estrutura da rede IPÊ de 2016 (slide 8) para 2020/2021?
    \item[Resposta] Aumento da capacidade agregada de 347Gb/s para 782 Gb/s, aumento da capacidade internacional de 116 Gb/s para 149 
    \item[2] Qual a diferença entre Web e Internet?
    \item[Resposta] A internet é o que conecta os computadores, enquanto a Web através do protocolo HTTP permite acessar o contéudo da internet através de páginas.
    \item[3] Quais orgãos administram o ponto br (.br) para além do slide 17?
    \item[Resposta] Comitê Gestor, Núcleo do ponto br, Registro de domínios, Reportes de incidentes de segurança
     ceptro.br, ix.br, registro.br
    \item[4] Quais características do protocolo HTTP descritas na RFC você já conhecia?
    \item[Resposta] Resource, entity, client, user agent, server, proxy, gateway, tunnel, cache
    \item[5] Qual o motivo de haver 2 chaves diferentes na figura do slide 32?
    \item[Resposta]Uma é a chave pública usada pra criptografar e a outra é a chave privada usada pra descriptografar.
\end{description}
\section{Aula 2}
\label{sec:Aula2}
\begin{description}
    \item[1] Por que “faltam” camadas no roteador e no switch do slide 7?
    \item[Resposta] Pelas funções do roteador e do switch, roteador tem camada de rede e o switch somente camada de Enlance.
    \item[2] Por que seu roteador Wi-Fi não é um roteador “de verdade”?
    \item[Resposta] Porque a tabela de roteamento tem apenas uma entrada. 
    \item[3] Qual a porta padrão dos seguintes protocolos: DHCP, HTTPS e POP3.
    \item[Resposta] DHCP: 67/68, HTTPS: 443, POP3: 995
    \item[4] Ainda há endereços IPv4 disponíveis no Brasil? Quando esgotaram ou quando
    esgotarão?
    \item[Resposta] Não, se esgotou em 19/08/2020
    \item[5] Qual o ip local da sua máquina? E da macalan?
    \item[Resposta]  192.168.0.15 /   200.17.202.6 
    \item[6] Qual a rota padrão da sua máquina? E da macalan?
    \item[Resposta]  192.168.0.1/ 200.17.202.62
    \item[7] Qual o caminho (route) mais comum entre sua máquina e a macalan?
    \item[Resposta]    
    Rastreando a rota para macalan.c3sl.ufpr.br [200.17.202.6]
    com no máximo 30 saltos:
    \item[1] 11 ms    10 ms    10 ms  192.168.0.1
    \item[2] 21 ms    19 ms    20 ms  10.53.160.1
    \item[3] 30 ms   152 ms   114 ms  bd040075.virtua.com.br [189.4.0.117]
    \item[4] 20 ms    28 ms    16 ms  bd040074.virtua.com.br [189.4.0.116]
    \item[5] 22 ms    23 ms    20 ms  as10881.curitiba.pr.ix.br [200.219.140.4]
    \item[6] 27 ms    37 ms    43 ms  p2-v103-araucaria-lapa.pop-pr.rnp.br [200.238.139.10]
    \item[7] 20 ms    18 ms    26 ms  200.17.202.62
    \item[8] 20 ms    18 ms    25 ms  macalan.c3sl.ufpr.br [200.17.202.6]
    
    \item[7] A partir de diferentes máquinas, o caminho (route) até a macalan muda?
    \item[Resposta]  Testei da Macalan para o Facebook e da minha máquina para o Facebook. 
    Sim, na minha máquina passou pela meu provedor de internet, por exemplo enquanto o da macalan passou pela rnp.

\end{description}
\end{document}