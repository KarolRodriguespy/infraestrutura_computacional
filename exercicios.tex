\documentclass[12pt]{article}
\usepackage[utf8]{inputenc}
\usepackage{lmodern}
\usepackage[T1]{fontenc}

\title{Exercicios Infraestrutura Computacional }
\author{Karoline Rodrigues}
\begin{document}
\normalsize
\maketitle
\section{Aula 1}
\label{sec:Aula1}
\begin{description}
    \item[1] Quais as diferenças na estrutura da rede IPÊ de 2016 (slide 8) para 2020/2021?
    \item[Resposta] Aumento da capacidade agregada de 347Gb/s para 782 Gb/s, aumento da capacidade internacional de 116 Gb/s para 149 
    \item[2] Qual a diferença entre Web e Internet?
    \item[Resposta] A internet é o que conecta os computadores, enquanto a Web através do protocolo HTTP permite acessar o contéudo da internet através de páginas.
    \item[3] Quais orgãos administram o ponto br (.br) para além do slide 17?
    \item[Resposta] Comitê Gestor, Núcleo do ponto br, Registro de domínios, Reportes de incidentes de segurança
     ceptro.br, ix.br, registro.br
    \item[4] Quais características do protocolo HTTP descritas na RFC você já conhecia?
    \item[Resposta] Resource, entity, client, user agent, server, proxy, gateway, tunnel, cache
    \item[5] Qual o motivo de haver 2 chaves diferentes na figura do slide 32?
    \item[Resposta]Uma é a chave pública usada pra criptografar e a outra é a chave privada usada pra descriptografar.
\end{description}
\section{Aula 2}
\label{sec:Aula2}
\begin{description}
    \item[1] Por que “faltam” camadas no roteador e no switch do slide 7?
    \item[Resposta] Pelas funções do roteador e do switch, roteador tem camada de rede e o switch somente camada de Enlance.
    \item[2] Por que seu roteador Wi-Fi não é um roteador “de verdade”?
    \item[Resposta] Porque a tabela de roteamento tem apenas uma entrada. 
    \item[3] Qual a porta padrão dos seguintes protocolos: DHCP, HTTPS e POP3.
    \item[Resposta] DHCP: 67/68, HTTPS: 443, POP3: 995
    \item[4] Ainda há endereços IPv4 disponíveis no Brasil? Quando esgotaram ou quando
    esgotarão?
    \item[Resposta] Não, se esgotou em 19/08/2020
\end{description}
\end{document}